% !mode::"TeX:UTF-8"
%!TEX program = xelatex
\documentclass[10pt]{beamer}
\usepackage{ctex} 
%\usepackage{beamerthemeshadow} %使用 shadow 风格

\usepackage{amsfonts}
\usepackage{amssymb}
\usepackage{amsmath}
\usepackage{amsthm}
\usepackage{bm} % Required for bold math symbols (used in the footer of the slides)
\usepackage{graphicx} % Required for including images in figures
\usepackage{tikz} % Required for colored boxes
\usetikzlibrary{matrix}
\usetikzlibrary{cd}
\usepackage{tikz-cd}
\usepackage{lastpage} % For printing the total number of pages at the bottom of each 
%\usepackage{tocstyle} % Required for customizing the table of contents


% There are many different themes available for Beamer. 
% http://deic.uab.es/~iblanes/beamer_gallery/index_by_theme.html
%\usetheme{AnnArbor}   %蓝色和黄色 
% \usetheme{Antibes}  %黑色蓝色 
%\usetheme{Bergen}
%\usetheme{Berkeley}
%\usetheme{Berlin}
%\usetheme{Boadilla}  %浅蓝色
%\usetheme{boxes}
%\usetheme{CambridgeUS}
%\usetheme{Copenhagen}
%\usetheme{Darmstadt}
%\usetheme{default}
%\usetheme{Frankfurt}
%\usetheme{Goettingen}  %黑蓝 没有logo
%\usetheme{Hannover}
%\usetheme{Ilmenau}
%\usetheme{JuanLesPins}
%\usetheme{Luebeck}
%\usetheme{Madrid}
%\usetheme{Malmoe}
%\usetheme{Marburg}
%\usetheme{Montpellier}
%\usetheme{PaloAlto}
%\usetheme{Pittsburgh}
%\usetheme{Rochester}
%\usetheme{Singapore}  %蓝色系  
%\usetheme{Szeged}
%\usetheme{Warsaw}

\setlength{\parindent}{0em}

\mode<presentation>
{
\usetheme{Malmoe}
\useinnertheme{rounded}
 \usecolortheme{orchid}
\usefonttheme{serif}
\setbeamercovered{transparent}  
}
% \setbeamertemplate{frametitle}
% {\vspace*{5pt}
%   \textbf{\insertframetitle}\\[-5pt]
%   {\color{magenta}\dotfill}
%   \vskip3pt\par
% }


\title{数量关系比例模块---工程问题}
\author{宁颖丹}
\date{2015年12月}

% \logo{\includegraphics[width=1.3cm,height=1.3cm]{logo-CAS}}
% Delete this, if you do not want the table of contents to pop up at
% the beginning of each subsection:
%\AtBeginSubsection[]
%{
%  \begin{frame}<beamer>{Outline}
%    \tableofcontents[currentsection]   %,currentsubsection
%  \end{frame}
%}

% Let's get started
\begin{document}
\small
\begin{frame}
  \titlepage
\end{frame}

% \begin{frame}{目录}
%   \tableofcontents
%   % You might wish to add the option [pausesections]
% \end{frame}

% Section and subsections will appear in the presentation overview
% and table of contents.
% \section{比例问题}
% \begin{frame}{比例问题}

%    工程问题
%    \newline
%    工程问题变形 


% \end{frame}

\section{工程问题}
\begin{frame}{工程问题}
关键: 设整或$1$
\[\mbox{工作量}=\mbox{工作效率}\times \mbox{时间}\]
 \begin{block}{类型}
  \begin{itemize}
    \item[1] 比例关系
    \item[2] 合作工作
    \item[3] 轮流工作 
  \end{itemize}
\end{block} 


\end{frame}
\begin{frame}{1.比例关系}
\begin{block}{例1}
  修一条公路,每人每天工作效率同,计划180人一年完成,工作4个月,特殊情况要求提前2个月完成,需要增加工人多少名?\\
  A.50\quad
  B.65\quad
  C.70\quad
  {\color{red}D.60}
\end{block}
\begin{alertblock}{解答}
  \[12-4=8 \quad   8-2=6\]
$$180\times \frac{8}{6}=240 \quad 240-180=60$$
\end{alertblock}
\end{frame}
\begin{frame}{2.合作工作}
\begin{block}{例2}
  (2014.75)甲乙共同完成A,B两个项目。甲单独完成A需13天,完成B需7天;乙单独完成A需11天,B需9天。若两对合作用最短时间完成两个项目,则最后一天两队需要共同工作多长时间就可以完成任务?\\
  A. 1/12天 \quad   B.1/9天\quad     C.1/7天\quad      {\color{red}D.1/6天}
\end{block}
\begin{center}
  \begin{tabular}{c|c|c}
 & A&B\\
 \hline
 甲&13天&7天\\
 乙&11天&9天
\end{tabular}
\end{center}

最短时间:甲完成B,乙完成A,再共同完成A.
\begin{alertblock}{解答}
设A总量为1,B结束后A的工作量还有4/11, 甲乙一起完成A效率为(1/13+1/11),
需要的时间为4/11/ (1/13+1/11)=13/6 最后一天需要共同工作1/6。
\end{alertblock}
\end{frame}
\begin{frame}{3.轮流工作}
\begin{block}{例3}
  一条隧道甲单独20天完成,乙单独要10天,如果甲先挖1天,乙接替甲挖1天,再由甲接替乙挖一天,两人如此交替工作。那么,挖完这条隧道共用多少天?\\
{\color{red}A.14}    \quad        B.16  \quad       C .15  \quad        D.13
\end{block}
\begin{alertblock}{解答}
总量为20,甲乙效率分别为1,2,每两天工作量为3,
$20=3\times6+1+1$,即 甲、乙先各干6天,然后甲干1天,剩下的工程量为1,乙用半天完成,总时间为$6\times2+1+1=14$天。

\end{alertblock}
\end{frame}

\section{工程问题变形}
\begin{frame}{工程问题变形}
\begin{block}{类型}
  \begin{itemize}
  \item[1] 水管问题
  \item[2] 牛吃草问题 
\end{itemize}
\end{block}
\end{frame}
\begin{frame}{1.水管问题}
\begin{block}{例4}
  同时打开泳池A、B两个水管,加满水需$1.5h$且A比B多进水$180m^3$,若单独打开A,加满水需2小时40分钟。则B管每分钟进水多少$m^3$?\\
   A.6  \quad  B.7  \quad  C.8  \quad D.9

\end{block}
\begin{alertblock}{解答}
90分钟A+90分钟B=160分钟A\\
A,B效率比90:70=9:7

\end{alertblock}
\end{frame}
\begin{frame}{2.牛吃草问题}
草不断生长速度不变,牛不断吃草切每头牛每天吃的量同,供不同数量牛吃,每天消耗草量就不同。
\newline
\newline
初始草量=(吃草速度1$-$草生长速度)$\times$时间1\\
 \quad \quad \quad \quad \      =(吃草速度2$-$草生长速度)$\times$时间2
\newline
\newline
草生长速度=(吃草速度1$\times$时间1$-$吃草速度2$\times$时间2)/(时间1$-$时间2)
\newline
\newline
初始草量=(吃草速度$-$草生长速度)$\times$时间
\end{frame}
\begin{frame}{2.牛吃草问题}
\begin{block}{例5}
  (2013.68)某河段沉积和沙供80人连续开采6个月或60人连续开采10个月,如果要保证改河段河沙不被开采枯竭,问最多可供多少人进行连续不间断的开采?(假设河段河沙沉积速度相对稳定) \\
   A.25     \quad   {\color{red}B.30}    \quad C.35    \quad  D.40
\end{block}
\begin{alertblock}{解答}
河沙每月沉积量$(60\times 10-80\times 6)/(10-6)=30$
开采量不能大过河沙沉积量,所以最多30人。
\end{alertblock}
\end{frame}
\section{总结}
\begin{frame}{总结}
工作总量$=\mbox{时间}_1\times \mbox{效率}_1+\mbox{时间}_2\times \mbox{效率}_2+\cdots+\mbox{时间}_n\times \mbox{效率}_n$
\newline
\newline
进水量=(进水速度$-$排水速度)$\times$ 时间
\newline
\newline
生长速度$=\frac{(\mbox{速度}_1\times \mbox{时间}_1-\mbox{速度}_2\times \mbox{时间}_2)}{(\mbox{时间}_1-\mbox{时间}_2)}$

\end{frame}



\begin{frame}

\begin{center}

  \Huge{谢谢大家!}
\end{center}

\end{frame}

\end{document}
